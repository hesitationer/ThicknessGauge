\section{Teknologivalg}
Denne sektion beskriver de forskellige delelementer projektet består af. Det er både med hensyn til hardware men hvilke udviklingsværktøj der blev benyttet. For at kunne danne et komplet billede, er der information medtaget omkring nogle af beslutninger og deres bevæggrunde. 

\subsection{Kamera}
Der er valgt at benytte sig af et kamera, der på baggrund af en laser der kaster en linje, kan måle forskel i højden fra hvor laserens linje kastes på en termoplastisk markering i forhold til, hvor den rammer ved siden af markeringen.
Kameraet er monokromatisk for at udnytte lyset bedre og har en opløsning på 5 megapixel.

\subsection{Laser}
Laseren kaster en linje ned på markeringsområdet med en styrke der gør det er muligt at kunne aflæse dens lokalisation. Forskellen på hvor laseren rammer på striben og hvor den ikke gør udgør den højde der er interessant.

\subsection{C++}
Udviklingen har indtil videre foregået i C$++$, da dette potentielt åbner op for at benytte sig af softwaren, eller dele af den, på generisk vis på tværs af diverse platforme, da softwaren er skrevet med udgangspunkt i standard template library.
Softwaren kræver en compiler der understøtter C++14, for en komplet liste over understøttede compilere, se \href{http://en.cppreference.com/w/cpp/compiler_support}{her}.

\subsection{Udviklingsværktøj}
Softwaren er udviklet ved brug af Microsoft Visual Studio, Community Edition. Selve koden skulle være 100\% kompatibel med andre udviklingskæder som f.eks. GCC. Det ville dog kræve at der blev sat et projekt op der sørger for de korrekte imports.

\subsection{OpenCV}
Til denne proces bliver Open Computer Vision \cite{OpenCV} benyttet, hvilket er et gratis computer vision system der gør det muligt at lave behandling på dataen for at lokalisere laserlinjen på de billeder kameraet tager.

\subsection{PvAPI}
Kameraet bliver kontrolleret ved hjælp af PvAPI. PvAPI er ellers blevet erstattet af det nyere API fra Allied Vision, Vimba. Men dette API kunne ikke bruges til projektet da det fejler med at kunne kontrollere elementære ting som f.eks. eksponering osv.
Dog skal Vimba's drivere til kameraet været installeret på systemet.

\subsection{Begrundelse}
Valget af kamera og laser er baseret på prisen, et moderne kamera med en tilstrækkelig kvalitet koster under 6.000,- DKr, og en linjelaser koster mellem 2.000,- DKr. og 3.000,- DKr. En laser scannings enhed koster over 15.000,- Dkr. hvilket vil sætte for meget pres på budgettet.

Både udviklingsværktøjet, opencv og pvapi er gratis, og kan benyttes i ubegrænset omfang, også i kommercielle forbindelser.
