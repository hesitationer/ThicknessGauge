\section{Triangulering\label{triangulering}}

Da højden vil blive målt i Y-aksen, og afstanden fra kameraet til målingspunktet vil være kendt på forhånd, vil afvigelser i højde for målepunkterne være nødt til at tage højde for forskelligheden.
De pixel positioner der findes for målingspunkterne er derfor allerede kendt i deres virkelige afstand fra fokus punktet.

\begin{figure}[!ht]
  \centering
    \includegraphics[scale=0.6]{Billeder/3D-Triangulation.jpg}
  \caption{Et eksempel på en 3D profil triangulerings opsætning\label{fig:triangulation1}}
\end{figure}

\begin{figure}[!ht]
  \centering
    \includegraphics[scale=0.6]{Billeder/1506VSDintegrationF1.jpg}
  \caption{Et eksempel der illustrerer at der kan måles højde\label{fig:triangulation2}}
\end{figure}

\subsection{Trianguleringsmetoder}

\begin{figure}[!ht]
  \centering
    \includegraphics[scale=0.6]{Billeder/1506VSDintegrationF2.jpg}
  \caption{Fire forskellige opsætningsmetoder for vision baseret triangulering\label{fig:triangulation3}}
\end{figure}

Ovenstående billede illustrerer nogle af de mest gængse opsætninger, der alle har deres fordele og ulemper.




