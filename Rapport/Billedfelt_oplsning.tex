\section{Billedfelt - Opløsning}

Denne sektion omhandler billedfeltet kontra opløsning, og den problematik der ligger bag.

\subsection{Information}

Som udgangspunkt vil der i den afstand der skal tages et billede fra være kendt. Derfor er det muligt at udregne hvor mange pixels der er pr. mm (eller cm). Denne udregning betyder at hvis alle forvrængninger af billedet vil kunne blive taget med i den endelige udregning for en eventuelt højdeforskel baseret ud fra målingspunkterne.

\subsection{\label{ref:kort}Kortlægning af pixels}

Ud fra de måleresultater der er fremkommet, er det muligt at kunne kortlægge de individuelle pixels til et antal mm. Ved en ru optælling af pixels kontra deres reale størrelse i millimeter, er der som udgangspunkt ca. 5 pixels pr. millimeter.
Da dette er en utilstrækkelig opløsning for at kunne opretholde kravet omkring præcisionen (ref blah blah), skal denne kortlægning foregå på subpixel niveau, ned til $\frac{1}{100}$ pixel.

