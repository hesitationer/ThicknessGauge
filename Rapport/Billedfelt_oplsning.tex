\section{Billedfelt - Opløsning}

Denne sektion omhandler billedfeltet kontra opløsning, og den problematik der ligger bag.

\subsection{Information}

Som udgangspunkt vil der i den afstand der skal tages et billede fra være kendt. Derfor er det muligt at udregne hvor mange pixels der er pr. mm (eller cm). Denne udregning betyder at hvis alle forvrængninger af billedet vil kunne blive taget med i den endelige udregning for en eventuelt højdeforskel baseret ud fra målingspunkterne.

\subsection{Resultat}

Det er på baggrund af informationerne derfor muligt at kunne lave rimelig korrekte udregninger der gør at hver måling bliver så god som mulig. Effekten af at have et mindre område med flere pixels vil kun være en fordel. Det er derfor en fordel at kunne have et kamera med høj opløsning og kvalitet. Så hvis det er muligt skal komprimeringen helt udelukkes eller være minimal da JPEG artifakter kun vil forringe kvaliteten og derved forværre målings præcisionen.
