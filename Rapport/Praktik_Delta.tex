%PraktikDelta
\section{DELTA}

\subsection{Lidt om}
DELTA Road Sensors er afdelingen hvor der udvikles produkter der er vejrelaterede. F.eks. skiltemåler, vejstribe RL målere både som håndholdt og mobil. Der eksisterer kunder verden over, og det er typisk instanser der tilsvarer det danske vejdirektorat der er kunder.

\subsection{Afdelingen}
Da både hardware- og sofwaredele bliver udviklet helt fra bunden, spiller softwareudviklerne en stor rolle i hele processen. Dette reflekteres i deres konstante involvering i processen samt at de typisk skal have et dybt kendskab til produktet. Grundet denne viden udøver de også support på produkterne, hvilket er en meget vigtig del af det samlede salg, da nogle af produkter ligger på omkring en million.

\subsection{Udviklingsmetoderne}
På afdelingen findes der ikke nogen forskrifter til hvorledes produkterne udvikles og hvilke værktøjer der skal bruges. De enkelte udviklere besidder fuld kontrol over hvorledes det skal foregå, hvilket giver en høj grad af fleksibilitet, men kræver at hver udvikler skal som udgangspunkt være kendt med alle de gængse metoder og sprog.

\subsection{Samarbejde med andre afdelinger}
Afdelingerne fungerer som enkelte instanser.
Det resulterer derfor i at de forskellige afdelinger tager betaling fra hinanden internt, for at holde styr på hvor meget tid der bliver brugt på hvad, og hvordan det ser ud økonomisk i forhold til budgettet. DELTA er meget bevidst omkring hvad de bruger deres budget, og sørger for at alt bliver dokumenteret og at de korrekte afdelinger "betaler" hvad de skal til hinanden af samme årsag.



