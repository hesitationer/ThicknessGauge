\section{Introduktion}

Denne rapport afdækker praktikforløbet hos Delta (Force, Hørsholm). Den beskriver de forskellige faser i projektet, samt belyser de problematikker der opstod i forbindelse med at løse opgaven.
Praktikopholdet hos Delta er baseret på et afklaringsprojekt med henblik på at få afdækket om hvorhvidt det er muligt at kunne måle højden af en vejstripe (termoplast). I den forbindelse var der fra starten en del problematik, bl.a. var der på forhånd ikke udlagt hvorledes dette skulle foregå. På den baggrund startede forløbet med at der skulle bruges tre uger på indsamling af information omkring hvilken form for teknologi der skulle benyttes. Valget endte med at stå mellem 3d laserskanning eller computer vision.

Det viste sig dog hurtigt at laserskanning var udelukket grundet de høje omkostninger det ville indebære, og der blev derfor fundet et ældre kamera frem der kunne benyttes til projektet.
